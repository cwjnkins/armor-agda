% Created 2023-04-07 Fri 19:22
% Intended LaTeX compiler: pdflatex
\documentclass[11pt]{article}
\usepackage[utf8x]{inputenc}
\usepackage[T1]{fontenc}
\usepackage{graphicx}
\usepackage{grffile}
\usepackage{longtable}
\usepackage{wrapfig}
\usepackage{rotating}
\usepackage[normalem]{ulem}
\usepackage{amsmath}
\usepackage{textcomp}
\usepackage{amssymb}
\usepackage{capt-of}
\usepackage{hyperref}
\usepackage{bbm}
\usepackage[greek, english]{babel}
\usepackage{latex/agda}
\DeclareUnicodeCharacter{7522}{\ensuremath { _i}}
\DeclareUnicodeCharacter{8337}{\ensuremath { _e}}
\DeclareUnicodeCharacter{8346}{\ensuremath { _p}}
\DeclareUnicodeCharacter{7523}{\ensuremath { _r}}
\DeclareUnicodeCharacter{8321}{\ensuremath { _1}}
\DeclareUnicodeCharacter{8322}{\ensuremath { _2}}
\DeclareUnicodeCharacter{955}{\ensuremath{\lambda}}
\DeclareUnicodeCharacter{8759}{\ensuremath{::}}
\DeclareUnicodeCharacter{737}{\ensuremath { ^l}}
\DeclareUnicodeCharacter{8799}{\ensuremath { \overset{?}{=}}}
\DeclareUnicodeCharacter{8252}{\ensuremath { !!}}
\DeclareUnicodeCharacter{8779}{\ensuremath { \cong}}
\author{Christa Jenkins}
\date{\today}
\title{AERES: Summary of verification architecture}
\hypersetup{
 pdfauthor={Christa Jenkins},
 pdftitle={AERES: Summary of verification architecture},
 pdfkeywords={},
 pdfsubject={},
 pdfcreator={Emacs 27.1 (Org mode 9.4.6)}, 
 pdflang={English}}
\begin{document}

\maketitle
\begin{code}[hide]
open import Aeres.Prelude
  hiding (Dec ; yes ; no ; Unique)
open import Aeres.Binary

module summary where

open Base256
\end{code}


\section{Operative Notions}
\label{sec:org7332b29}

\begin{itemize}
\item \textbf{soundness} If the parser accepts the string, so does the grammar

\item \textbf{completeness} If the grammar accepts the string, so does the parser

\item \textbf{secure completeness} If the grammar accepts the string, so does the parser,
and there are no two distinct ways for the grammar to accept the string.

NOTE TO OMAR: I'm not sure if this is good terminology, or even if it is a
good idea to group completeness (a relation between the grammar and parser)
and uniqueness (a property of the grammar).
\end{itemize}

\section{Overview}
\label{sec:orgedf3fce}

\begin{enumerate}
\item The Aeres external driver is invoked with the filepath of the certificate
chain we wish to check.
The driver invokes Aeres with the contents of this file.

\item Aeres uses its verified PEM parser library to parse the PEM certificate
chain, then decodes the Base64-encoded certificates into a single
bytestring.\footnote{We maybe could have decoded it to a list of bytestrings and
parsed each, come to think of it\ldots{}}

(Sound and complete parsing)

\item Aeres uses its verified X.509 parser library to parse the bytestring into a
list of certificates.

(Sound, complete, secure)

\item Aeres then checks several semantic properties not suitable for expressing
in the grammar (e.g., validity period of cert contains current time)

\item For each cert, Aeres outputs the bytestring serializations for the TBS
certificate, signature, and public key, and also outputs the signature
algorithm OIDLeastBytes

\item The external driver verifies the public key signatures.
\end{enumerate}



\section{Design (Challenges and Solutions)}
\label{sec:org4bf85da}
\subsection{Grammar}
\label{sec:org29db303}
\textbf{Challenge} Our first and most fundamental question is: how shall we represent
the grammar?
Recall that our operative notion of soundness is "if the parser accepts the
string, then so does the grammar."
We also wish for our formulation of the grammar to serve as a readable
formalization of the X.509 and X.690 specification.

\textbf{Solution} In general purpose functional languages, inductive types are a
natural choice for expressing the grammar of a language.
Our choice of formalizing X.509 and X.680 is \emph{inductive families}, the
generalization of inductive types to a dependently typed setting.

Let us consider a simple example: X.690 DER Boolean values.
The BER require that Boolean values consists of a single octet
with \texttt{FALSE} represented by the setting all bits to 0, and the DER further
stipulates that \texttt{TRUE} be represented by setting all bits to 1.
We represent these constraints as follows.

\begin{code}
module BoolExample where
  data BoolRep : Bool → UInt8 → Set where
    falseᵣ : BoolRep false (UInt8.fromℕ 0)
    trueᵣ  : BoolRep true (UInt8.fromℕ 255)
  record BoolValue (@0 bs : List UInt8) : Set where
    constructor mkBoolValue
    field
      v     : Bool
      @0 b  : UInt8
      @0 vᵣ : BoolRep v b
      @0 bs≡ : bs ≡ [ b ]
\end{code}

\begin{enumerate}
\item First, we define a binary relation \AgdaDatatype{BoolRep} that relates Agda
\AgdaDatatype{Bool} values to the octet values specified by X.690 DER
(\AgdaFunction{UInt8.fromℕ} converts a non-negative unbounded integer to its
\AgdaFunction{UInt8} representation, provided Agda can verify automatically
the given number is less than 256).

\item Next, we define a record \AgdaDatatype{BoolValue} for the representation of
the X.690 Boolean value itself.

\begin{itemize}
\item Each production rule of the grammar, such as \AgdaDatatype{BoolValue}, is
represented by a type family of type
\AgdaSymbol{@}\AgdaSymbol{0}\AgdaSpace{}\AgdaDatatype{List}\AgdaSpace{}\AgdaDatatype{UInt8}\AgdaSpace{}\AgdaSymbol{→}\AgdaSpace{}\AgdaPrimitive{Set},
which we interpret as the type of predicates over byte-strings (we will
explain the \AgdaSymbol{@}\AgdaSymbol{0} business shortly).

\item The fields of the record are the Boolean value \AgdaField{v}, its
byte-string representation \AgdaField{b}, a proof of type
\AgdaDatatype{BoolRep}\AgdaSpace{}\AgdaField{v}\AgdaSpace{}\AgdaField{b}
that \AgdaField{b} is the correct representation of \AgdaField{b}, and a
proof that the byte-string representation of this terminal of the grammar
is the singleton list consisting of \AgdaField{b} (written \AgdaFunction{[}\AgdaSpace{}\AgdaField{b}\AgdaSpace{}\AgdaFunction{]})
\end{itemize}
\end{enumerate}


The \AgdaSymbol{@}\AgdaSymbol{0} annotations on types and fields indicate that
the values are \emph{erased at run-time.}
We do this for two reasons: to reduce the space and time overhead for
executions of Aeres, and to serve as machine-enforced documentation
delineating the parts of the formalization that are purely for the purposes of
verification.
\subsection{Parser}
\label{sec:org2821fe9}

\textbf{Challenge:} Next, we must design the parser.
We desire that the parser by sound and complete \emph{by construction}.

\textbf{Solution:} For our parser to be sound, when it succeeds we have it return a
proof that the byte-string conforms to grammar. For completeness, when it
fails we have it return a \emph{refutation} --- a proof that there is no possible
way for the grammar to accept the given byte-string.
The two of these together are captured nicely by the notion of
\emph{decidability}, formalized in the Agda standard library as \AgdaDatatype{Dec}
(we show a simplified, more intuitive version of this type below)
\begin{code}
module DecSimple where
  data Dec (P : Set) : Set where
    yes : P → Dec P
    no  : ¬ P → Dec P
\end{code}

Let us examine (a slightly simplified version of) the definition of
\AgdaDatatype{Parser} used in Aeres.
Below, module parameter \AgdaBound{S} is the type of the characters of the
alphabet over which we have defined a grammar.

\begin{code}
module ParserSimple (S : Set) where
  record Success (@0 A : List S → Set) (@0 xs : List S) : Set where
    constructor success
    field
      @0 prefix : List S
      read   : ℕ
      @0 read≡ : read ≡ length prefix
      value  : A prefix
      suffix : List S
      @0 ps≡    : prefix ++ suffix ≡ xs
  record Parser (M : Set → Set) (@0 A : List S → Set) : Set where
    constructor mkParser
    field
      runParser : (xs : List S) → M (Success A xs)
  open Parser public
\end{code}

\begin{itemize}
\item We first must specify what the parser returns when it succeeds.
This is given by the record \AgdaDatatype{Success}.

\begin{itemize}
\item Parameter \AgdaBound{A} is the production rule (e.g.,
\AgdaDatatype{BoolValue}), and \AgdaBound{xs} is the generic-character
string which we parsed.
Both are marked erased from run-time

\item Field \AgdaField{prefix} is the prefix of our input string consumed by
the parser.
We do not need to keep this at run-time, however for the purposes of
length-bounds checking we do keep its length \AgdaField{read} available
at run-time.

\item Field \AgdaField{value} is a proof that the prefix conforms to the
production rule \AgdaBound{A}.

\item Field \AgdaField{suffix} is what remains of the string after parsing.
We of course need this at run-time to continue parsing any subsequent
production rules.

\item Finally, field \AgdaField{ps≡} relates \AgdaField{prefix} and
\AgdaField{suffix} to the string \AgdaBound{xs} that we started with,
i.e., they really are a prefix and suffix of the input.
\end{itemize}

\item Next, we define the type \AgdaDatatype{Parser} for parsers.
\begin{itemize}
\item Parameter \AgdaBound{M} is used to give us some flexibility in the type
of the values returned by the parser.
Almost always, it is instantiated with
\AgdaDatatype{Logging}\AgdaSpace{}\AgdaFunction{∘}\AgdaSpace{}\AgdaDatatype{Dec},
where \AgdaDatatype{Logging} provides us lightweight debugging
information.
Parameter \AgdaBound{A} is, again, the production rule we wish to parse.

\item \AgdaDatatype{Parser} consists of a single field \AgdaField{runParser},
which is a dependently type function taking a character string
\AgdaBound{xs} and returning
\AgdaBound{M}\AgdaSpace{}\AgdaSymbol{(}\AgdaDatatype{Success}\AgdaSpace{}\AgdaBound{A}\AgdaSpace{}\AgdaBound{xs}\AgdaSymbol{)}
(again, usually \AgdaDatatype{Logging}\AgdaSpace{}\AgdaSymbol{(}\AgdaDatatype{Dec}\AgdaSpace{}\AgdaSymbol{(}\AgdaDatatype{Success}\AgdaSpace{}\AgdaBound{A}\AgdaSpace{}\AgdaBound{xs}\AgdaSymbol{)}\AgdaSymbol{)})
\end{itemize}
\end{itemize}

\subsubsection{Example}
\label{sec:orgf89fca6}

It is helpful to see an example parser.

    \begin{AgdaAlign}
    \begin{code}[hide]
module BoolParseExample where
  open import Aeres.Data.X690-DER.Boool.Properties
  open import Aeres.Data.X690-DER.Boool.TCB
  import      Aeres.Grammar.Definitions
  import      Aeres.Grammar.Parser
  open Aeres.Grammar.Definitions UInt8
  open Aeres.Grammar.Parser      UInt8
  open Aeres.Prelude
    using (Dec ; yes ; no)
    \end{code}

    \begin{code}
  private
    here' = "X690-DER: Bool"
  
  parseBoolValue : Parser (Logging ∘ Dec) BoolValue
  runParser parseBoolValue [] = do  {- 1 -}
    tell $ here' String.++ ": underflow"
    return ∘ no $ λ where
      (success prefix read read≡ value suffix ps≡) →
        contradiction (++-conicalˡ _ suffix ps≡) (nonempty value)
  runParser parseBoolValue (x ∷ xs) {- 2 -}
    with x ≟ UInt8.fromℕ 0 {- 3 -}
  ... | yes refl =
    return (yes (success _ _ refl (mkBoolValue _ _ falseᵣ refl) xs refl))
  ... | no x≢0
    with x ≟ UInt8.fromℕ 255 {- 3 -}
  ... | yes refl =
    return (yes (success _ _ refl (mkBoolValue _ _ trueᵣ refl) xs refl))
  ... | no  x≢255 = do {- 4 -}
    tell $ here' String.++ ": invalid boolean rep"
    return ∘ no $ λ where
      (success prefix _ _ (mkBoolValue v _ vᵣ refl) suffix ps≡) → ‼
        (case vᵣ of λ where
          falseᵣ → contradiction (∷-injectiveˡ (sym ps≡)) x≢0
          trueᵣ  → contradiction (∷-injectiveˡ (sym ps≡)) x≢255)
    \end{code}
    \end{AgdaAlign}

\begin{enumerate}
\item When the input string is empty, we emit an error message, then return a
proof that there is no parse of a \AgdaDatatype{BoolValue} for the empty
string

We use Agda's \AgdaKeyword{do}-notation to sequence our operations

\item If there is at least one character, we check
\item whether it is equal to 0 or 255.
If so, we affirm that this conforms to the grammar.
\item If it is not equal to either, we emit an error message then return a
parse refutation: to conform to \AgdaDatatype{BoolValue}, our byte must
be either 0 or 255.
\end{enumerate}

\subsubsection{Backtracking}
\label{sec:orgb054f08}

Although backtracking is not required to parse X.509, our parser has been
implemented with some backtracking to facilitate the formalization.
For an example, the X.690 specification for \AgdaDatatype{DisplayText}
states it may comprise an \AgdaDatatype{IA5String},
\AgdaDatatype{VisibleString}, \AgdaDatatype{VisibleString}, or
\AgdaDatatype{UTF8String}.
In the case that the give byte-string does not conform to \AgdaDatatype{DisplayTex}
providing a refutation is easier when we have direct evidence that it fails
to conform to each of these.

    \begin{AgdaAlign}
    \begin{code}[hide]
module DisplayTextExample where
  open import Aeres.Data.X509.DisplayText.TCB
  open import Aeres.Data.X509.IA5String
  open import Aeres.Data.X509.Strings
  open import Aeres.Data.X690-DER.TLV
  import      Aeres.Grammar.Parser 
  open Aeres.Grammar.Parser UInt8
  open Aeres.Prelude
    using (Dec ; yes ; no)
    \end{code}
    \begin{code}
  private
    here' = "X509: DisplayText"
  
  parseDisplayText : Parser (Logging ∘ Dec) DisplayText
  runParser parseDisplayText xs = do
    no ¬ia5String ← runParser (parseTLVLenBound 1 200 parseIA5String) xs
      where yes b → return (yes (mapSuccess (λ {bs} → ia5String{bs}) b))
    no ¬visibleString ← runParser (parseTLVLenBound 1 200 parseVisibleString) xs
      where yes b → return (yes (mapSuccess (λ {bs} → visibleString{bs}) b))
    no ¬bmp ← runParser (parseTLVLenBound 1 200 parseBMPString) xs
      where yes b → return (yes (mapSuccess (λ {bs} → bmpString{bs}) b))
    no ¬utf8 ← runParser (parseTLVLenBound 1 200 parseUTF8String) xs
      where yes u → return (yes (mapSuccess (λ {bs} → utf8String{bs}) u))
    return ∘ no $ λ where
      (success prefix read read≡ (ia5String x) suffix ps≡) →
        contradiction (success _ _ read≡ x _ ps≡) ¬ia5String
      (success prefix read read≡ (visibleString x) suffix ps≡) →
        contradiction (success _ _ read≡ x _ ps≡) ¬visibleString
      (success prefix read read≡ (bmpString x) suffix ps≡) →
        contradiction (success _ _ read≡ x _ ps≡) ¬bmp
      (success prefix read read≡ (utf8String x) suffix ps≡) →
        contradiction (success _ _ read≡ x _ ps≡) ¬utf8
      \end{code}
      \end{AgdaAlign}


\section{Complete and Secure Parsing}
\label{sec:org0321583}

Completeness of the parser is by construction, and straightforward to explain:
given a byte-string, if it conforms to the grammar then the parser accepts the
byte-string. The heart of the proof is proof by contradiction (which is
constructively valid, since the parser is itself evidence that conformance to
the grammar is decidable): suppose the parser rejects a string which conforms
to the grammar. Then, this rejection comes with a refutation of the
possibility that the string conforms with the grammar, contradicting our
assumption.

When it comes to security, we also care that the grammar is \emph{unambiguous},
by which we mean that there is at most one way in which the grammar might be
parsed.
This is formalized as \AgdaFunction{UniqueParse} below

  \begin{AgdaAlign}
  \begin{code}[hide]
module CompSec (S : Set) where
  open import Aeres.Grammar.Parser S
  open Aeres.Prelude
    using (Dec ; yes ; no)
  \end{code}
  \begin{code}
  Unique : Set → Set
  Unique P = (p₁ p₂ : P) → p₁ ≡ p₂

  UniqueParse : (List S → Set) → Set
  UniqueParse A = ∀ {@0 xs} → Unique (Success A xs)
  \end{code}

  We have a lemma that establishes a sufficient condition for when
  \AgdaFunction{UniqueParse} holds, whose premises are stated only in terms of
  properties of the grammar itself.
  These properties are:
  1. Any two witness[fn::By which we mean inhabitants of a type, when we interpret that type as a proposition under the Curry-Howard isomorphism]
     that a given string conforms to the grammar are equal (\AgdaFunction{Unambiguous}), and
  2. If two prefixes of the same string conform to the grammar, those prefixes are equal (\AgdaFunction{NonNesting})


  \begin{code}
  Unambiguous : (A : List S → Set) → Set
  Unambiguous A = ∀ {xs} → Unique (A xs)

  NonNesting : (A : List S → Set) → Set
  NonNesting A =
    ∀ {xs₁ ys₁ xs₂ ys₂}
    → (prefixSameString : xs₁ ++ ys₁ ≡ xs₂ ++ ys₂)
    → (a₁ : A xs₁) (a₂ : A xs₂) → xs₁ ≡ xs₂

  module _ {A : List S → Set} (uA : Unambiguous A) (nnA : NonNesting A) where
    @0 uniqueParse : UniqueParse A
    uniqueParse p₁ p₂
    {- = ... -}
    \end{code}
    \begin{code}[hide]
      with ‼ nnA (trans (Success.ps≡ p₁) (sym (Success.ps≡ p₂))) (Success.value p₁) (Success.value p₂)
    ... | refl
      with ‼ Lemmas.++-cancel≡ˡ (Success.prefix p₁) _ refl (trans (Success.ps≡ p₁) (sym (Success.ps≡ p₂)))
    ... | refl
      with ‼ (trans (Success.read≡ p₁) (sym (Success.read≡ p₂)))
    ... | refl
      with ‼ ≡-unique (Success.read≡ p₂) (Success.read≡ p₁)
      |    ‼ ≡-unique (Success.ps≡ p₂) (Success.ps≡ p₁)
    ... | refl | refl
      with ‼ uA (Success.value p₁) (Success.value p₂)
    ... | refl = refl
  \end{code}

  This finally brings us to the statement and proof of complete and secure parsing.
  \begin{code}
  Yes_And_ : {P : Set} → Dec P → (P → Set) → Set
  Yes (yes pf) And Q = Q pf
  Yes (no ¬pf) And Q = ⊥

  CompleteParse : (A : List S → Set) → Parser Dec A → Set
  CompleteParse A p =
    ∀ {bs} → (v : Success A bs) → Yes (runParser p bs) And (v ≡_)

  module _ {A : List S → Set}
    (uA : Unambiguous A) (nnA : NonNesting A) (parser : Parser Dec A)
    where
    @0 completeParse : CompleteParse A parser
    completeParse{bs} v
      with runParser parser bs
    ... | (yes v')  = uniqueParse uA nnA v v'
    ... | no ¬v     = contradiction v ¬v
  \end{code}
  \end{AgdaAlign}

\begin{enumerate}
\item We define an auxiliary predicate \AgdaFunction{Yes\_And\_} over decisions,
expressing that the decision is \AgdaInductiveConstructor{yes} and
the predicate \AgdaBound{Q} holds for the affirmative proof that comes with
it.
\item The predicate \AgdaFunction{CompleteParse} is defined in terms of
\AgdaFunction{Yes\_And\_}, expressing that if \AgdaBound{v} is a witness that
some prefix of \AgdaBound{bs} conforms to \AgdaBound{A}, then the parser
returns in the affirmative and the witness it returns is equal to
\AgdaBound{v}.

\item We then prove the property \AgdaFunction{CompleteParse} under the
assumption that \AgdaBound{A} is \AgdaFunction{Unambiguous} and
\AgdaFunction{NonNesting}.

The proof proceeds by cases on the result of running the parser on the
given string.
\begin{itemize}
\item If the parser produces an affirmation, we appeal to lemma
\AgdaFunction{uniqueParse}.
\item If the parser produces a refutation, we have a
\AgdaFunction{contradiction}
\end{itemize}
\end{enumerate}


\section{Semantic Checks}
\label{sec:orgcf94a3f}

Some properties that we wish to verify are not as suitable for formalization
as part of the grammar.
For example, the X.509 specification requires that the signature algorithm
field of the TBS certificate matches the signature algorithm listed in the
outer certificate --- a highly non-local property.
Aeres checks such properties after parsing.
For each of these, we first write a specification of the property, then a
proof that that property is \emph{decidable}.
This proof is itself the function that we call to check whether the property
holds, and interpreted as such, it is sound and complete by construction for
the same reasons that our parser is.

  \begin{AgdaAlign}
\begin{code}[hide]
module SemanticExample where
  open Aeres.Prelude using (Dec ; yes ; no)
  open import Aeres.Data.X509
  import      Aeres.Grammar.Definitions
  open Aeres.Grammar.Definitions UInt8
\end{code}
\begin{code}
  SCP1 : ∀ {@0 bs} → Cert bs → Set
  SCP1 c = Cert.getTBSCertSignAlg c ≡ Cert.getCertSignAlg c

  scp1 :  ∀ {@0 bs} (c : Cert bs) → Dec (SCP1 c)
  scp1 c =
    case (proj₂ (Cert.getTBSCertSignAlg c) ≋? proj₂ (Cert.getCertSignAlg c)) ret (const _) of λ where
      (yes ≋-refl) → yes refl
      (no ¬eq) → no λ where refl → contradiction ≋-refl ¬eq
\end{code}
  \end{AgdaAlign}
\end{document}